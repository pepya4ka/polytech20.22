\documentclass[14pt,a4paper,report]{report}
\usepackage[a4paper, mag=1000, left=2.5cm, right=1cm, top=2cm, bottom=2cm, headsep=0.7cm, footskip=1cm]{geometry}
\usepackage[utf8]{inputenc}
\usepackage[english,russian]{babel}
\usepackage{indentfirst}
\usepackage[dvipsnames]{xcolor}
\usepackage[colorlinks]{hyperref}
\usepackage{listings} 
\usepackage{fancyhdr}
\usepackage{caption}
\usepackage{graphicx}
\hypersetup{
	colorlinks = true,
	linkcolor  = black
}

\usepackage{titlesec}
\titleformat{\chapter}
{\Large\bfseries} % format
{}                % label
{0pt}             % sep
{\huge}           % before-code


\DeclareCaptionFont{white}{\color{white}} 

% Listing description
\usepackage{listings} 
\DeclareCaptionFormat{listing}{\colorbox{gray}{\parbox{\textwidth}{#1#2#3}}}
\captionsetup[lstlisting]{format=listing,labelfont=white,textfont=white}
\lstset{ 
	% Listing settings
	inputencoding = utf8,			
	extendedchars = \true, 
	keepspaces = true, 			  	 % Поддержка кириллицы и пробелов в комментариях
	language = C,            	 	 % Язык программирования (для подсветки)
	basicstyle = \small\sffamily, 	 % Размер и начертание шрифта для подсветки кода
	numbers = left,               	 % Где поставить нумерацию строк (слева\справа)
	numberstyle = \tiny,          	 % Размер шрифта для номеров строк
	stepnumber = 1,               	 % Размер шага между двумя номерами строк
	numbersep = 5pt,              	 % Как далеко отстоят номера строк от подсвечиваемого кода
	backgroundcolor = \color{white}, % Цвет фона подсветки - используем \usepackage{color}
	showspaces = false,           	 % Показывать или нет пробелы специальными отступами
	showstringspaces = false,    	 % Показывать или нет пробелы в строках
	showtabs = false,           	 % Показывать или нет табуляцию в строках
	frame = single,              	 % Рисовать рамку вокруг кода
	tabsize = 2,                  	 % Размер табуляции по умолчанию равен 2 пробелам
	captionpos = t,             	 % Позиция заголовка вверху [t] или внизу [b] 
	breaklines = true,           	 % Автоматически переносить строки (да\нет)
	breakatwhitespace = false,   	 % Переносить строки только если есть пробел
	escapeinside = {\%*}{*)}      	 % Если нужно добавить комментарии в коде
}

\begin{document}

\def\contentsname{Содержание}

% Titlepage
\begin{titlepage}
	\begin{center}
		\textsc{ФЕДЕРАЛЬНОЕ ГОСУДАРСТВЕННОЕ АВТОНОМНОЕ ОБРАЗОВАТЕЛЬНОЕ УЧРЕЖДЕНИЕ ВЫСШЕГО ОБРАЗОВАНИЯ
«САНКТ-ПЕТЕРБУРГСКИЙ ПОЛИТЕХНИЧЕСКИЙ УНИВЕРСИТЕТ ПЕТРА ВЕЛИКОГО»\\[5mm]
			Институт компьютерных наук и технологий\\
			Высшая школа интеллектуальных систем и суперкомпьютерных технологий}
		
		\vfill
		
		\textbf{Дисциплина «Цифровые ресурсы в научных исследованиях»\\[10mm]
		ОТЧЕТ\\[2mm]
		по лабораторной работе №1\\[2mm]
		на тему\\[2mm]
			«Первичный полнотекстовый поиск источников информации в глобальной сети интернет»\\[41mm]
		}
	\end{center}
	
	\hfill
	\begin{minipage}{.4\textwidth}
		Выполнил:\\[2mm] 
        Студент группы 3540901/02001\\
		Дроздов Н.Д.\\[2mm]
        «....» ................... 2020г., ...............\\
        \hspace*{38mm} (Подпись)\\[2mm]
		Проверил:\\[2mm] 
		Бендерская Е.Н.\\[2mm]
		«....» ................... 2020г., ...............\\
        \hspace*{38mm} (Подпись)\\
	\end{minipage}
	\vfill
	\begin{center}
		Санкт-Петербург\\ \the\year\ г.
	\end{center}
\end{titlepage}

% Contents
\tableofcontents
\clearpage

\chapter{Лабораторная работа №1}

\section{Цель работы}

Дать ответы на поставленные вопросы.

\section{Программа работы}

\begin{enumerate}
	\item Сформулировать основные ключевые слова и словосочетания для выполнения первичного информационного запроса на русском языке по теме аналитического отчета. Записать ключевые слова и словосочетания в порядке релевантности. Выполнить поиск по всем ключевым словам и словосочетаниям, используемым как в одном поисковом запросе, так и по-отдельности, а провести поиск непосредственно по названию темы аналитического отчета. Провести первичную оценку результатов поиска путем выставления оценки от 1 до 10 степени релевантности полученных результатов для каждого запроса и представить их в отчете.
	
	\item Оценить качество и количество представленных ссылок для лучшего варианта поискового запроса - оценка степени удовлетворенности качеством представленных источников от 1 (совсем не соответствует ожиданиям) до 10 (именно то, что ожидалось найти) и оценка достаточности представленных источников от 1 (очень мало) до 10 (очень много), и на этом основании выполнить уточнение информационного запроса – дополнение ключевых слов (исключение ключевых слов), расширение ключевых словосочетаний синонимами и т.д. Провести первичную оценку результатов поиска путем выставления оценки от 1 до 10 степени релевантности полученных результатов для каждого запроса и представить их в отчете.
	
	\item Сравнить набор ключевых слов из лучших найденных источников (представить в отчете) и использованных при выполнении поискового запроса. Сделать вывод о необходимости коррекции состава ключевых слов или формулировки общего поискового запроса.
	
	\item Выполнить п. 1-3 в еще одной (другой) поисковой системе. Сравнить полученные результаты и сформулировать выводы.
	
	\item Изучить возможности поисковых систем по работе с расширенным поиском. Использовать расширенный поиск – привести примеры выполненного расширенного поиска и оценить качество полученных результатов по сравнению с простым поисковым запросом. Сделать выводы.
	
	\item Выбрать из найденных источников наиболее релевантный и осуществить переработку первичного текста во вторичный, путем выполнения аннотирования с критической оценкой первоисточника и реферирования. Составить описательную аннотацию к тексту(описательная аннотация излагает, «о чем» написан первоисточник и лишь называет основные моменты содержания), составить реферативную аннотацию к тексту (реферативная аннотация, отражая основные вопросы содержания, в предельно сжатом виде передает также выводы по каждому из затронутых вопросов и по материалу в целом), составить аннотацию-резюме к тексту (аннотация-резюме характеризуется четкой передачей главного тематического содержания в предельно сжатом виде). Составить информативный реферат текста (не более 1/2 - 3/4 страницы)
	
	\item Выполнить п. 1-6 для информационного запроса на английском языке.
	
	\item Представить отчет о выполненной работе.
	
\end{enumerate}

\clearpage

\section{Ход работы}

\subsection{Задание 1}

\subsubsection{Сформулировать основные ключевые слова и словосочетания для выполнения первичного информационного запроса на русском языке по теме аналитического отчета. Записать ключевые слова и словосочетания в порядке релевантности. Выполнить поиск по всем ключевым словам и словосочетаниям, используемым как в одном поисковом запросе, так и по-отдельности, а провести поиск непосредственно по названию темы аналитического отчета. Провести первичную оценку результатов поиска путем выставления оценки от 1 до 10 степени релевантности полученных результатов для каждого запроса и представить их в отчете.}

Основные ключевые слова и словосочетания для выполнения первичного информационного запроса на русском языке по теме "Большие данные и искусственный интеллект: новое и хорошо забытое старое" являются:
\begin{itemize}
	\item Большие данные и искусственный интеллект;
	\item Интеллектуальный анализ данных;
	\item Обработка больших данных;
	\item Методы интеллектуального анализа данных;
	\item Задачи интеллектуального анализа данных;
	\item Области применения методов и технологий интеллектуального анализа данных;
	\item Области применения методов и технологий обработки больших данных;
	\item Большие данные;
	\item Искусственный интеллект;
	\item Машинное обучение.
\end{itemize}
%интеллектуальные системы и помощники (концепция чат бота, кортрана, сири, яндекс и т.д. Зачем, почему? Перспективы телеграмм)

Оценка результатов поиска:\\

\begin{tabular}{ | l | l | }
\hline
Большие данные и искусственный интеллект & 8 \\ \hline
Интеллектуальный анализ данных & 10 \\ \hline
Обработка больших данных & 9 \\ \hline
Методы интеллектуального анализа данных & 9\\ \hline
Задачи интеллектуального анализа данных & 5 \\ \hline
Области применения методов и технологий интеллектуального анализа данных & 8 \\ \hline
Области применения методов и технологий обработки больших данных & 6 \\ \hline
Большие данные & 9 \\ \hline
Искусственный интеллект & 7 \\ \hline
Машинное обучение & 7 \\
\hline
\end{tabular}



\subsection{Задание 2}

\subsubsection{Оценить качество и количество представленных ссылок для лучшего варианта поискового запроса - оценка степени удовлетворенности качеством представленных источников от 1 (совсем не соответствует ожиданиям) до 10 (именно то, что ожидалось найти) и оценка достаточности представленных источников от 1 (очень мало) до 10 (очень много), и на этом основании выполнить уточнение информационного запроса – дополнение ключевых слов (исключение ключевых слов), расширение ключевых словосочетаний синонимами и т.д. Провести первичную оценку результатов поиска путем выставления оценки от 1 до 10 степени релевантности полученных результатов для каждого запроса и представить их в отчете.}

По итогам оценки результатов поиска можно выделить наилучший вариант поискового запроса - Интеллектуальный анализ данных.
Проведем оценку качества и количества представленных ссылок для данного поискового запроса:\\

\begin{tabular}{ l l }
Интеллектуальный анализ данных \\
Качество & 10 \\
Количество & 11665 (7) \\
\end{tabular}\\

После уточнения информационного запроса я могу сказать, что лучшим запросом является оригинал, потому что с помощью этого запроса можно добиться наилучшего результата при поиске.
Были предложены следующие варианты поискового запроса:
\begin{itemize}
	\item Интеллектуальный анализ больших данных;
	\item Интеллектуальная обработка больших данных;
	\item Искусственный интеллект в области обработки больших данных.
\end{itemize}

Оценка предложенных поисковых запросов:\\

\begin{tabular}{ | l | l | }
\hline
Интеллектуальный анализ больших данных & 6 \\ \hline
Интеллектуальная обработка больших данных & 6 \\ \hline
Искусственный интеллект в области обработки больших данных & 5 \\
\hline
\end{tabular}


\subsection{Задание 3}

\subsubsection{Сравнить набор ключевых слов из лучших найденных источников (представить в отчете) и использованных при выполнении поискового запроса. Сделать вывод о необходимости коррекции состава ключевых слов или формулировки общего поискового запроса.}

После сравнения набора ключевых слов из лучших найденных источников(например [1] и [2]) с использованными при выполнении поискового запроса, можно заметить, что чаще всего первые ключевые слова полностью совпадают с текстом поискового запроса.

\subsection{Задание 4}

\subsubsection{Выполнить п. 1-3 в еще одной (другой) поисковой системе. Сравнить полученные результаты и сформулировать выводы}

В качестве второй поисковой системы было предложено использовать научную электронную библиотеку elibrary [4].

Оценка результатов поиска:\\

\begin{tabular}{ | l | l | }
\hline
Большие данные и искусственный интеллект & 5 \\ \hline
Интеллектуальный анализ данных & 10 \\ \hline
Обработка больших данных & 7 \\ \hline
Методы интеллектуального анализа данных & 6\\ \hline
Задачи интеллектуального анализа данных & 4 \\ \hline
Области применения методов и технологий интеллектуального анализа данных & 6 \\ \hline
Области применения методов и технологий обработки больших данных & 6 \\ \hline
Большие данные & 7 \\ \hline
Искусственный интеллект & 7 \\ \hline
Машинное обучение & 7 \\
\hline
\end{tabular}\\

Можно заметить, что во второй поисковой системе запрос "Интеллектуальный анализ данных" также имеет наивысшую оценку.

Проведем оценку качества и количества представленных ссылок для выбранного поискового запроса:\\

\begin{tabular}{ l l }
Интеллектуальный анализ данных \\
Качество & 9 \\
Количество & 617243 (5) \\
\end{tabular}\\

Уточнение поискового запроса не зависит от поисковой системы, поэтому, я считаю, что наилучшим поисковым запросом из предложенных является "Интеллектуальный анализ данных". 

\subsection{Задание 5}

\subsubsection{Изучить возможности поисковых систем по работе с расширенным поиском. Использовать расширенный поиск – привести примеры выполненного расширенного поиска и оценить качество полученных результатов по сравнению с простым поисковым запросом. Сделать выводы}

Анализ поисковых запросов проводился в научной электронной библиотеке «КИБЕРЛЕНИНКА». В ней присутствует возможность фильтрации результатов при поиске. Перечень доступных фильтров:
\begin{itemize}
	\item Фильтр по году;
	\item Фильтр по терму OECD;
	\item Фильтр по научным базам;
	\item Фильтр по журналам.
\end{itemize}

Для выполнения заданий я использовал фильтр по терму OECD, а именно "Компьютерные и информационные науки". Можно было бы также использовать фильтр по году и поставить 2019-2020 года, так как в последние годы искусственный интеллект развивается все больше и больше, а значит качество научных публикаций на эту тему также растет.
Используя фильтрацию можно получить результат более точный или более свежий, что сильно упрощает поиск.



\subsection{Задание 6}

\subsubsection{Выбрать из найденных источников наиболее релевантный и осуществить переработку первичного текста во вторичный, путем выполнения аннотирования с критической оценкой первоисточника и реферирования. Составить описательную аннотацию к тексту(описательная аннотация излагает, «о чем» написан первоисточник и лишь называет основные моменты содержания), составить реферативную аннотацию к тексту (реферативная аннотация, отражая основные вопросы содержания, в предельно сжатом виде передает также выводы по каждому из затронутых вопросов и по материалу в целом), составить аннотацию-резюме к тексту (аннотация-резюме характеризуется четкой передачей главного тематического содержания в предельно сжатом виде). Составить информативный реферат текста (не более 1/2 - 3/4 страницы)}

В качестве наиболее релевантного источника было использовать научную статью [1].

В данной статье излагаются особенности применения технологии интеллектуального анализа данных, описываются основные инструменты реализации данного метода анализа и представляются направления их развития.

В начале статьи делают утверждение о том, что нынешние учетные системы, которые используются на каждом предприятии, плохо приспособлены для принятия решения. В связи с этим, для анализа данных, которые накоплены в данных системах, есть необходимость использовать интеллектуальный анализ данных.

После чего автор статьи объясняет что такое интеллектуальный анализ данных и описывает, грубо говоря, стадии жизненного цикла анализа.

В статье также затронута тема реализации интеллектуального анализа. Представлены примеры механизмов, серверов, инструментов и платформ.

В конце статьи автор делает выводы по статье, говоря об направлениях развития. Он выделяет три направления, немного описывая каждое из них.


\subsection{Задание 7}

\subsubsection{Выполнить п. 1-6 для информационного запроса на английском языке}

В качестве поисковой системы было предложено использовать бесплатную поисковую систему Google Scholar [6].

Основные ключевые слова и словосочетания для выполнения первичного информационного запроса на английском языке по теме "Большие данные и искусственный интеллект: новое и хорошо забытое старое" являются:
\begin{itemize}
	\item Big data and artificial intelligence;
	\item Data mining;
	\item Big data processing;
	\item Big data;
	\item Artificial intelligence;
	\item Machine learning;
	\item Information analysis.
\end{itemize}

Оценка результатов поиска:\\

\begin{tabular}{ | l | l | }
\hline
Big data and artificial intelligence & 8 \\ \hline
Data mining & 10 \\ \hline
Big data processing & 9 \\ \hline
Big data & 9\\ \hline
Artificial intelligence & 5 \\ \hline
Machine learning & 7 \\ \hline
Information analysis & 7 \\
\hline
\end{tabular}\\

По итогам оценки результатов поиска можно выделить наилучший вариант поискового запроса - Data mining.

Проведем оценку качества и количества представленных ссылок для данного поискового запроса:\\

\begin{tabular}{ l l }
Data mining \\
Качество & 10 \\
Количество & примерно 3750000 (7) \\
\end{tabular}\\

Если сравнить ключевые слова в публикациях, полученных в результате поиска, с выбранным поисковым запросом, можно сделать вывод, что чаще всего выбранная тема попадает в первые ключевые слова.\\
Вследствие чего можно сделать вывод, что Data mining является хорошим поисковым запросом.

В Google Scholar также имеется возможность фильтрации и сортировки с целью получения конкретного результата.
Основными фильтрами и сортировками являются:
\begin{itemize}
	\item Фильтр по году;
	\item Сортировка по дате;
	\item Сортировка по релеватности;
\end{itemize}

Также стоит отметить функцию "Похожие запросы", которая может сузить спектр поискового запроса.

В качестве наиболее релевантного источника было использовать научную статью [5].

В предложенной статье автор рассказывает про Big Data и развитие сбора данных в современном мире.

В этой статье представлена теорема HACE, которая характеризует особенности революции больших данных, и предлагает модель обработки больших данных с точки зрения интеллектуального анализа данных.

Эта модель, управляемая данными, включает в себя объединение источников информации по запросу, интеллектуальный анализ и анализ, моделирование интересов пользователей, а также соображения безопасности и конфиденциальности.


\subsection{Задание 8}

\subsubsection{Представить отчет о выполненной работе}


\section{Вывод}
В ходе проделанной лабораторной работы были составлены ключевые слова, основные поисковые запросы на русском и английском языках на тему "Большие данные и искусственный интеллект: новое и хорошо забытое старое". Были проанализированы различные поисковые системы с составленными поисковыми запросами. Стоит отметить возможность фильтрации в поисковых системах, что сильно ускоряет и упрощает поиск необходимой информации.\\
Благодаря ключевым словам в научных публикациях, можно видоизменять свой поисковый запрос, что также может привести к упрощению поиска нужной информации.

\clearpage

\section{Список литературы}

\begin{flushleft}

[1] ТЕХНОЛОГИИ РЕАЛИЗАЦИИ ИНТЕЛЛЕКТУАЛЬНОГО АНАЛИЗА ДАННЫХ [Научная статья/Электронный ресурс]. — URL: https://cyberleninka.ru/article/n/tehnologii-realizatsii-intellektualnogo-analiza-dannyh-1

[2] ИНТЕЛЛЕКТУАЛЬНЫЙ АНАЛИЗ ДАННЫХ И ОБЛАЧНЫЕ ВЫЧИСЛЕНИЯ [Научная статья/Электронный ресурс]. — URL: https://cyberleninka.ru/article/n/intellektualnyy-analiz-dannyh-i-oblachnye-vychisleniya

[3] НАУЧНАЯ ЭЛЕКТРОННАЯ БИБЛИОТЕКА «КИБЕРЛЕНИНКА» [Электронный ресурс]. — URL: https://cyberleninka.ru/

[4] НАУЧНАЯ ЭЛЕКТРОННАЯ БИБЛИОТЕКА «elibrary.ru» [Электронный ресурс]. — URL: https://elibrary.ru/

[5] Data mining with big data [Научная статья/Электронный ресурс]. — URL: https://ieeexplore.ieee.org/abstract/document/6547630

[6] Google Scholar [Электронный ресурс]. — URL: https://scholar.google.com/

\end{flushleft}
	
\end{document}